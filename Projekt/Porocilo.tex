\documentclass[a4paper,11pt]{article}
\usepackage{a4wide}
\usepackage{fullpage}
\usepackage[utf8x]{inputenc}

%\usepackage[light,math]{anttor}
\usepackage[T1]{fontenc}

\usepackage[slovene]{babel}
%\selectlanguage{slovene}
\usepackage[toc,page]{appendix}
\usepackage[pdftex]{graphicx} 

\usepackage{lmodern}
\usepackage{amsmath}
\usepackage{amssymb}
\usepackage{amsthm}
\usepackage{amsfonts}
\usepackage{mathtools}
\usepackage{enumitem}
\usepackage{amsfonts}
\usepackage{amsmath}
\usepackage{setspace}
\usepackage{relsize}
\usepackage{color}
\definecolor{light-gray}{gray}{0.95}
\usepackage{listings} 
\usepackage{hyperref}
\renewcommand{\baselinestretch}{1.2} 
\renewcommand{\appendixpagename}{Priloge}

\lstset{ 
language=Matlab,
basicstyle=\footnotesize,
basicstyle=\ttfamily\footnotesize\setstretch{1},
backgroundcolor=\color{light-gray},
}

%\usepackage{algorithm}
%\usepackage[noend]{algpseudocode}

\theoremstyle{definition} 
\newtheorem*{definicija}{Definicija}
\newtheorem*{trditev}{Trditev}


\theoremstyle{plain} 
\newtheorem*{izrek}{Izrek}
\newtheorem*{posledica}{Posledica}
\newtheorem*{zgled}{Zgled}
\newtheorem{primer}{Primer}

\title{Krožnica in ostale stožnice \\
v racionalni B\'ezierjevi obliki
}
\author{Sara Bizjak in Urša Blažič}
\date{\today}

%%%%%%%%%%%%%%%%%%%%%%%%%%%%%%%%%%%%%%%%%%%

\begin{document}

\maketitle

\section{Povzetek}

\section{Uvod}

\section{Racionalne B\'ezierjeve krivulje}

Racionalna B\'ezierjeva krivulja stopnje $n$ v $\mathbb{R}^d$ je projekcija polinomske B\'ezierjeve krivulje stopnje $n$ v $\mathbb{R}^{d+1}$ na hiperravnino $w=1$, kjer točko v  $\mathbb{R}^{d+1}$  označimo z
$$(\boldsymbol{x},w)=(x_1,x_2,\dots,x_d,w).$$
Projekcija je definirana kot
$$(\boldsymbol{x},w)\mapsto (\frac{1}{w}\boldsymbol{x},1).$$
Točke oblike $\lambda(\boldsymbol{x}, w)$ za $\lambda\neq 0$ se preslikajo v isto točko na projektivni ravnini, točke z
$w = 0$ pa predstavljajo točke v neskončnosti.

\begin{definicija}
\emph{Racionalna B\'ezierjeva krivulja} stopnje $n$ je podana s parametrizacijo $\boldsymbol{r}:[0,1]\rightarrow \mathbb{R}^d$, določeno s predpisom
$$\boldsymbol{r}(t)=\frac{\mathlarger{\sum}_{i=0}^n w_i\boldsymbol{b}_iB_i^n(t)}{\mathlarger{\sum}_{i=0}^n w_iB_i^n(t)},$$
kjer so $\boldsymbol{b}_i$ kontrolne točke krivulje, $w_i\in\mathbb{R}^d$ uteži, $B_i^n$ pa $i$-ti Bernsteinov bazni polinom stopnje~$n$.
\end{definicija}

Uteži predstavljajo dodatne proste parametre pri oblikovanju. Krivuljo lahko brez škode za splošnost reparametriziramo tako, da sta $w_0$ in $w_n$ enaka $1$, ostale uteži pa so prosti parametri. Taki obliki pravimo \emph{standardna oblika} racionalne B\'ezierjeve krivulje.

\section{Stožnice v racionalni B\'ezierjevi obliki}
Stožnice bomo zapisali s pomočjo krivulje stopnje $2$, zato bo kontrolni poligon sestavljen iz treh kontrolni točk. Ker lahko izberemo uteži v standardni obliki, velja: $$w_0=w_3=1,\,w_2=w.$$
Stožnice lahko zapišemo v racionalni B\'ezierjevi obliki kot 
$$\boldsymbol{r}(t)=\frac{\boldsymbol{b}_0\cdot B_0^2+w\cdot\boldsymbol{b}_1\cdot B_1^2+\boldsymbol{b}_2\cdot B_2^2}{ B_0^2+w\cdot B_1^2+ B_2^2}\,\, t\in[0,1],$$
kjer so $\boldsymbol{b}_0, \boldsymbol{b}_1, \boldsymbol{b}_2 \in \mathbb{R}^2$ kontrolne točke krivulje, $w$ utež, vezana na kontrolno točko $\boldsymbol{b}_1$, $B_i^2,\,i=~0,1,2,$ pa Bernsteinovi bazni polinomi:
\begin{align*}
B_0^2(t) &= (1-t)^2 \\
B_1^2(t) &= 2t\cdot(1-t) \\
B_2^2(t) &= t^2 \\
\end{align*}

%With this representation,the type of conic is characterized by the value of the middle weight,w:r(t)is an ellipse whenw<1, aparabola whenw=1 and a hyperbola whenw>1. Also, whenwis negative,r(t)is the complementarysegment of the original conic segment (see (Lee, 1987) and (Farin, 1993))


\section{Racionalna B\'ezierjeva krivulja za krožnico}
Krožnico lahko opišemo kot racionalno B\'ezierjevo krivuljo $C(t)=(X(t),Y(t))$ s pomočjo projekcije krivulje $\tilde{C}(t)=(\tilde{X}(t), \tilde{Y(t)}, W(t))$ na ravnino $w=1$. 
Enačbo krožnice v $\mathbb{R}^2$ lahko zapišemo kot
$$X(t)^2+Y(t)^2=1.$$
Koordinate točk zamenjamo s koordinatami prostora $\mathbb{R}^3$, ki smo jih dobili s projekcijo na ravnino $w=1$, da dobimo
$$\left(\frac{\tilde{X}(t)}{W(t)}\right)^2+\left(\frac{\tilde{Y}(t)}{W(t)}\right)^2=1$$
$$\Rightarrow \tilde{X}(t)^2+\tilde{Y}(t)^2-W(t)^2=0.$$
Vidimo, da enačba predstavlja enačbo stožca. Torej, s projekcijo krivulje, ki leži na stožcu, na ravnino $w=1$ dobimo krožnico $C(t)$.
SLIKA, MATLAB?
tle še manjka \\

Krožnico bi radi opisali z racionalno B\'ezierjevo krivuljo in ne z zlepkom krivulj. Poiskali bomo najmanjšo stopnjo krivulje, s katero lahko opišemo krožnico. 

\subsection{ Kvadratična krivulja}
Pokazali bomo, da racionalna kvadratična krivulja ne more opisati celotne krožnice. Vse neracionalne kvadratične B\'ezierjeve krivulje so parabole, vse parabole pa dobimo kot presek stožca z ravnino, ki je vzporedna eni od nosilk stožca. Ko ravnino, s katero presekamo stožec, premikamo proti nosilki stožca, opazimo, da bo krožni lok v projekciji  na ravnino opisal vedno večji kot, torej smo vedno bližje polnemu krogu. Ko pa naredimo presek ravnine z nosilko stožca, je presek premica, ki se v projekciji slika v eno točko. Krožnice zato ne moremo zapisati s kvadratično racionalno krivuljo. 

Opazimo, da lahko s parabolo zapišemo krožne loke, ki jih lahko opišemo z naslednjim kontrolnim poligonom:
\begin{align*}
\boldsymbol{\tilde{P}}_0 &= (cos(\theta), -sin(\theta), 1)\\
\boldsymbol{\tilde{P}}_1 &= (1, 0, cos(\theta))\\
\boldsymbol{\tilde{P}}_2 &= (cos(\theta), sin(\theta), 1),
\end{align*}
kjer je $\theta$ polovični kot krožnega loka. Če izberemo samo pozitivne uteži ($\cos{\theta}>0$), je $C(t)$ manj kot $180^{\circ}$, če pa je srednja utež enaka $0$ ($\cos{\theta}=0$), je $C(t)$ točno $180^{\circ}$. ??
% Za prvo in zadnjo točko si izberemo točki, ki ležita na krožnici, srednjo točko pa zaradi simetrije izberemo na polovičnem kotu med njima. Pri tem izberemo take uteži, da zadoščajo enačbi stožca. 

\subsection{Kubična krivulja}
Pokazali bomo, da tudi racionalna kubična krivulja ne more opisati celotne krožnice. \\
Za kontrolni poligon potrebujemo štiri kontrolne točke. S krivuljo interpoliramo prvo in zadnjo kontrolno točko, izberemo ju na stožcu.
????

\subsection{Krivulja 4.~stopnje}
Da bomo dobili vse krivulje, v enačbo za stožec $\tilde{X}(t)^2+\tilde{Y}(t)^2-W(t)^2=0$ vstavimo ??? \\
Bernsteinove bazne polinome $B_i^8(t),\,i=0,\ldots,8$ enačimo z $0$ in dobimo devet enačb.
Brez škode za splošnost vzamemo $\boldsymbol{\tilde{P}}_0 =\boldsymbol{\tilde{P}}_4 = (1,0,1)$. Ker $\boldsymbol{\tilde{P}}_1$ in $\boldsymbol{\tilde{P}}_3$ ležita v tangentni ravnini $\boldsymbol{\tilde{P}}_0$, velja $\tilde{x}_1=w_1$ in $\tilde{x}_3=w_3$ (saj imamo ravnino $x=w$). Devet enačb se nam tako reducira v pet enačb:
\begin{align*}
\tilde{y}_3 &=- \tilde{y}_1 \\
\tilde{x}_3 &= - \tilde{x}_1 \\
3\tilde{x}_2 + 4\tilde{y}_1^2 - 3w_2 &= 0 \\
\tilde{x}_1\tilde{x}_2 + \tilde{y}_1\tilde{y}_2  - \tilde{x}_1w_2 &= 0 \\
9\tilde{x}_2^2 - 8\tilde{y}_1^2 + 9\tilde{y}_2^2 - 9w_2^2&= 0 \\
\end{align*}
Iz zadnjih treh enačb dobimo dve možni rešitvi
\begin{align*}
\tilde{y}_1 &= \alpha \\
\tilde{x}_2 &=-\frac{3w_2-4\tilde{x}_1^2+2}{3}\\
\tilde{y}_2 &= \frac{4}{3}\tilde{x}_1\alpha
\end{align*}
in
\begin{align*}
\tilde{y}_1 &= -\alpha \\
\tilde{x}_2 &=-\frac{3w_2-4\tilde{x}_1^2+2}{3}\\
\tilde{y}_2 &= -\frac{4}{3}\tilde{x}_1\alpha
\end{align*}

kjer je $\alpha=(\frac{3w_2}{2}-\tilde{x}_1^2+\frac{1}{2})^{\frac{1}{2}}$

Dobimo naslednje kontrolne točke, ki tvorijo kontrolni poligon:
\begin{align*}
\boldsymbol{\tilde{P}}_0 &= (1,0,1) \\
\boldsymbol{\tilde{P}}_1 &= (\tilde{x}_1,\pm\alpha,\tilde{x}_1) \\
\boldsymbol{\tilde{P}}_2 &= (-\frac{3w_2-4\tilde{x}_1^2+2}{3},\pm\frac{4}{3}\tilde{x}_1\alpha,w_2) \\
\boldsymbol{\tilde{P}}_3 &= (-\tilde{x}_1,\mp\alpha,-\tilde{x}_1) \\
\boldsymbol{\tilde{P}}_4 &= (1,0,1)
\end{align*}

Ker mora biti $\alpha$ pozitivno število, mora veljati
$$w_2>-\frac{1}{3}$$
in
$$-\left(\frac{3w_2+1}{2}\right)^{\frac{1}{2}}<\tilde{x}_1<\left(\frac{3w_2+1}{2}\right)^{\frac{1}{2}}$$
Vidimo, da zaradi $\boldsymbol{\tilde{P}}_1$ in $\boldsymbol{\tilde{P}}_3$ nikoli ne dobimo kvartične B\'ezierjeve krivulje z vsemi pozitivnimi utežmi. \\
Če izberemo $\tilde{x}_1=0$, dobimo  
\begin{align*}
\boldsymbol{\tilde{P}}_0 &= (1,0,1) \\
\boldsymbol{\tilde{P}}_1 &= (0,\pm (\frac{1}{2}+\frac{3}{2}w_2)^{1/2},0) \\
\boldsymbol{\tilde{P}}_2 &= (-\frac{2}{3}-w_2,0,w_2) \\
\boldsymbol{\tilde{P}}_3 &= (0,\mp(\frac{1}{2}+\frac{3}{2}w_2)^{1/2},0) \\
\boldsymbol{\tilde{P}}_4 &= (1,0,1)
\end{align*}
V tem primeru nimamo več negativnih uteži, dobimo pa dve uteži, ki sta enaki $0$. Povedali smo že, da točka z $w=0$ pri projekciji na ravnino $w=1$ predstavlja točko v neskončnosti. Torej, imamo dve točki v neskončnosti, tega pa pri implementaciji ne želimo. \\
Kako bi se znebili negativnih uteži? Tako, da dani krivulji dvignemo stopnjo. \\%takrat se bo naš kontrolni poligon pomaknil bližje h krivulji.
Pogoji, da ima taka krivulja same pozitivne uteži, so
$$-\frac{1}{4}<\tilde{x}_1<\frac{1}{4}$$
in
$$-\frac{3}{2}w_2<\tilde{x}_1<\frac{3}{2}w_2.$$

TUKI BI BLO FAJN DAT ŠE KAK PRIMER - SO V ČLANKU


\section{Kubični B\'ezierjev krožni lok}


\begin{thebibliography}{99}
\bibitem{chou}
J.~J.~Chou, \emph{Higher order Bezier circles}, Computer-Aided Design \textbf{27 (4)} (1995) 303--309

\bibitem{farin}
G.~Farin, \emph{Curves and surfaces for CAGD}, A Practical Guide, 5th ed., Morgan Kaufmann, 2002, poglavje 12.7

\bibitem{knez}
M.~Knez, J.~Grošelj, \emph{Računalniško podprto geometrijsko oblikovanje}, ... (2020)
\end{thebibliography}


\end{document}